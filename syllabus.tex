% Options for packages loaded elsewhere
% Options for packages loaded elsewhere
\PassOptionsToPackage{unicode}{hyperref}
\PassOptionsToPackage{hyphens}{url}
\PassOptionsToPackage{dvipsnames,svgnames,x11names}{xcolor}
%
\documentclass[
  letterpaper,
  DIV=11,
  numbers=noendperiod]{scrartcl}
\usepackage{xcolor}
\usepackage{amsmath,amssymb}
\setcounter{secnumdepth}{-\maxdimen} % remove section numbering
\usepackage{iftex}
\ifPDFTeX
  \usepackage[T1]{fontenc}
  \usepackage[utf8]{inputenc}
  \usepackage{textcomp} % provide euro and other symbols
\else % if luatex or xetex
  \usepackage{unicode-math} % this also loads fontspec
  \defaultfontfeatures{Scale=MatchLowercase}
  \defaultfontfeatures[\rmfamily]{Ligatures=TeX,Scale=1}
\fi
\usepackage{lmodern}
\ifPDFTeX\else
  % xetex/luatex font selection
\fi
% Use upquote if available, for straight quotes in verbatim environments
\IfFileExists{upquote.sty}{\usepackage{upquote}}{}
\IfFileExists{microtype.sty}{% use microtype if available
  \usepackage[]{microtype}
  \UseMicrotypeSet[protrusion]{basicmath} % disable protrusion for tt fonts
}{}
\makeatletter
\@ifundefined{KOMAClassName}{% if non-KOMA class
  \IfFileExists{parskip.sty}{%
    \usepackage{parskip}
  }{% else
    \setlength{\parindent}{0pt}
    \setlength{\parskip}{6pt plus 2pt minus 1pt}}
}{% if KOMA class
  \KOMAoptions{parskip=half}}
\makeatother
% Make \paragraph and \subparagraph free-standing
\makeatletter
\ifx\paragraph\undefined\else
  \let\oldparagraph\paragraph
  \renewcommand{\paragraph}{
    \@ifstar
      \xxxParagraphStar
      \xxxParagraphNoStar
  }
  \newcommand{\xxxParagraphStar}[1]{\oldparagraph*{#1}\mbox{}}
  \newcommand{\xxxParagraphNoStar}[1]{\oldparagraph{#1}\mbox{}}
\fi
\ifx\subparagraph\undefined\else
  \let\oldsubparagraph\subparagraph
  \renewcommand{\subparagraph}{
    \@ifstar
      \xxxSubParagraphStar
      \xxxSubParagraphNoStar
  }
  \newcommand{\xxxSubParagraphStar}[1]{\oldsubparagraph*{#1}\mbox{}}
  \newcommand{\xxxSubParagraphNoStar}[1]{\oldsubparagraph{#1}\mbox{}}
\fi
\makeatother


\usepackage{longtable,booktabs,array}
\usepackage{calc} % for calculating minipage widths
% Correct order of tables after \paragraph or \subparagraph
\usepackage{etoolbox}
\makeatletter
\patchcmd\longtable{\par}{\if@noskipsec\mbox{}\fi\par}{}{}
\makeatother
% Allow footnotes in longtable head/foot
\IfFileExists{footnotehyper.sty}{\usepackage{footnotehyper}}{\usepackage{footnote}}
\makesavenoteenv{longtable}
\usepackage{graphicx}
\makeatletter
\newsavebox\pandoc@box
\newcommand*\pandocbounded[1]{% scales image to fit in text height/width
  \sbox\pandoc@box{#1}%
  \Gscale@div\@tempa{\textheight}{\dimexpr\ht\pandoc@box+\dp\pandoc@box\relax}%
  \Gscale@div\@tempb{\linewidth}{\wd\pandoc@box}%
  \ifdim\@tempb\p@<\@tempa\p@\let\@tempa\@tempb\fi% select the smaller of both
  \ifdim\@tempa\p@<\p@\scalebox{\@tempa}{\usebox\pandoc@box}%
  \else\usebox{\pandoc@box}%
  \fi%
}
% Set default figure placement to htbp
\def\fps@figure{htbp}
\makeatother

\ifLuaTeX
  \usepackage{luacolor}
  \usepackage[soul]{lua-ul}
\else
  \usepackage{soul}
\fi




\setlength{\emergencystretch}{3em} % prevent overfull lines

\providecommand{\tightlist}{%
  \setlength{\itemsep}{0pt}\setlength{\parskip}{0pt}}



 


\KOMAoption{captions}{tableheading}
\makeatletter
\@ifpackageloaded{caption}{}{\usepackage{caption}}
\AtBeginDocument{%
\ifdefined\contentsname
  \renewcommand*\contentsname{Table of contents}
\else
  \newcommand\contentsname{Table of contents}
\fi
\ifdefined\listfigurename
  \renewcommand*\listfigurename{List of Figures}
\else
  \newcommand\listfigurename{List of Figures}
\fi
\ifdefined\listtablename
  \renewcommand*\listtablename{List of Tables}
\else
  \newcommand\listtablename{List of Tables}
\fi
\ifdefined\figurename
  \renewcommand*\figurename{Figure}
\else
  \newcommand\figurename{Figure}
\fi
\ifdefined\tablename
  \renewcommand*\tablename{Table}
\else
  \newcommand\tablename{Table}
\fi
}
\@ifpackageloaded{float}{}{\usepackage{float}}
\floatstyle{ruled}
\@ifundefined{c@chapter}{\newfloat{codelisting}{h}{lop}}{\newfloat{codelisting}{h}{lop}[chapter]}
\floatname{codelisting}{Listing}
\newcommand*\listoflistings{\listof{codelisting}{List of Listings}}
\makeatother
\makeatletter
\makeatother
\makeatletter
\@ifpackageloaded{caption}{}{\usepackage{caption}}
\@ifpackageloaded{subcaption}{}{\usepackage{subcaption}}
\makeatother
\usepackage{bookmark}
\IfFileExists{xurl.sty}{\usepackage{xurl}}{} % add URL line breaks if available
\urlstyle{same}
\hypersetup{
  pdftitle={Syllabus},
  colorlinks=true,
  linkcolor={blue},
  filecolor={Maroon},
  citecolor={Blue},
  urlcolor={Blue},
  pdfcreator={LaTeX via pandoc}}


\title{Syllabus}
\author{}
\date{}
\begin{document}
\maketitle


\subsection{Course info}\label{course-info}

\subsubsection{Class meetings}\label{class-meetings}

\begin{longtable}[]{@{}
  >{\raggedright\arraybackslash}p{(\linewidth - 4\tabcolsep) * \real{0.2131}}
  >{\raggedright\arraybackslash}p{(\linewidth - 4\tabcolsep) * \real{0.4426}}
  >{\raggedright\arraybackslash}p{(\linewidth - 4\tabcolsep) * \real{0.3443}}@{}}
\toprule\noalign{}
\endhead
\bottomrule\noalign{}
\endlastfoot
\textbf{Lecture} & Mon \& Wed 1:35 - 2:50pm & Cruzen-Murray Library
(CML) 208 \\
\textbf{Office Hours} & Mon 3:30 - 4:30pm & Boone 126B \\
\textbf{Office Hours} & Tue 10:20 - 11:20am & Boone 126B \\
\textbf{Office Hours} & Thu 1:30 - 3:30pm & Boone 126B \\
\end{longtable}

Office hours are also available by appointment, just email me!

\subsubsection{Instructor Information}\label{instructor-information}

\begin{itemize}
\tightlist
\item
  \textbf{Instructor}: Professor Eric Friedlander
\item
  \textbf{Office}: Boone 126B
\item
  \textbf{Email}:
  \href{mailto:efriedlander@collegeofidaho.edu}{\nolinkurl{efriedlander@collegeofidaho.edu}}
\end{itemize}

\subsection{Course Learning
Objectives}\label{course-learning-objectives}

By the end of the semester, you will be able to\ldots{}

\begin{itemize}
\tightlist
\item
  analyze real-world data to answer questions about multivariable
  relationships.
\item
  use R to fit and evaluate linear and logistic regression models.
\item
  assess whether a proposed model is appropriate and describe its
  limitations.
\item
  use Quarto to write reproducible reports.
\item
  effectively communicate statistical results through writing and oral
  presentations.
\end{itemize}

\subsection{Course community}\label{course-community}

\subsubsection{College of Idaho Honor
Code}\label{college-of-idaho-honor-code}

\begin{quote}
The College of Idaho maintains that academic honesty and integrity are
essential values in the educational process. Operating under an Honor
Code philosophy, the College expects conduct rooted in honesty,
integrity, and understanding, allowing members of a diverse student body
to live together and interact and learn from one another in ways that
protect both personal freedom and community standards. Violations of
academic honesty are addressed primarily by the instructor and may be
referred to the
\href{https://collegeofidaho.smartcatalogiq.com/en/current/Undergraduate-Catalog/Policies-and-Procedures/Academic-Misconduct}{Student
Judicial Board}.
\end{quote}

By participating in this course, you are agreeing that all your work and
conduct will be in accordance with the College of Idaho Honor Code.

\subsubsection{Disability Accommodation
Statement}\label{disability-accommodation-statement}

The College of Idaho seeks to provide an educational environment that is
accessible to the needs of students with disabilities. The College
provides reasonable services to enrolled students who have a documented
permanent or temporary physical, psychological, learning, intellectual,
or sensory disability that qualifies the student for accommodations
under the Americans with Disabilities Act or section 504 of the
Rehabilitation Act of 1973. If you have, or think you may have, a
disability that impacts your performance as a student in this class, you
are encouraged to arrange support services and/or accommodations through
the Department of Accessibility and Learning Excellence located in
McCain 201B and available via email at
\href{mailto:accessibility@collegeofidaho.edu}{\nolinkurl{accessibility@collegeofidaho.edu}}.
Reasonable academic accommodations may be provided to students who
submit appropriate and current documentation of their disability.
Accommodations can be arranged only through this process and are not
retroactively applied. More information can be found on the DALE webpage
(\url{https://www.collegeofidaho.edu/accessibility}).

\subsubsection{Communication}\label{communication}

All lecture notes, assignment instructions, an up-to-date schedule, and
other course materials may be found on the course website,
\href{https://math2025fa25.netlify.app}{math2025fa25.netlify.app}.

Periodic announcements will be sent via email and will also be available
through Canvas and grades will be stored in the Canvas gradebook. Please
check your email regularly to ensure you have the latest announcements
for the course.

\subsubsection{In class agreements}\label{in-class-agreements}

If we discuss/agree to something in class or office hours which requires
action from me (e.g.~``you may turn in your homework late due to a
sporting event''), you MUST send me a follow-up message. If you don't, I
will almost certainly forget, and our agreement will be considered null
and void.

\subsubsection{Getting help in the
course}\label{getting-help-in-the-course}

\begin{itemize}
\tightlist
\item
  If you have a question during lecture or lab, feel free to ask it!
  There are likely other students with the same question, so by asking
  you will create a learning opportunity for everyone.
\item
  I am here to help you be successful in the course. You are encouraged
  to attend \emph{office hours} to ask questions about the course
  content and assignments. Many questions are most effectively answered
  as you discuss them with others, so office hours are a valuable
  resource. You are encouraged to use them!
\item
  Outside of class and office hours, any general questions about course
  content or assignments should be posted on the class discussion forum
  on the
  \href{https://teams.microsoft.com/l/channel/19\%3AvBRIeFRpxyUJfvl-0Ur9bQWxhR_Wzkpr6iM6V_OBVKA1\%40thread.tacv2/?groupId=ffcfc61b-032c-4fe3-883b-fe9fd969ea51}{Teams
  Discussion Forum}. There is a chance another student has already asked
  a similar question, so please check the other posts before adding a
  new question. If you know the answer to a question posted in the
  discussion forum, you are encouraged to respond!
\end{itemize}

\subsubsection{Email}\label{email}

If you have questions about assignment extensions or accommodations,
please email
\href{mailto:efriedlander@collegeofidaho.edu}{\nolinkurl{efriedlander@collegeofidaho.edu}}.
Please see \hyperref[late-work-policy]{Late work policy} for more
information. \textbf{If you email me, please include ``MATH 2025'' in
the subject line.} Barring extenuating circumstances, I will respond to
MATH 2025 emails within 48 hours Monday - Friday. Response time may be
slower for emails sent Friday evening - Sunday.

Check out the \href{support.qmd}{Support} page for more resources.

\subsection{Textbook}\label{textbook}

The official textbook for this course is:

\begin{itemize}
\tightlist
\item
  Stat2: Modeling with Regression and ANOVA, 2nd ed.~by Cannon et al.
\end{itemize}

In addition, readings may be assigned from the following texts (all
freely available online).

\begin{itemize}
\item
  \href{https://r4ds.had.co.nz/}{R for Data Science} by Garret Grolemund
  and Hadley Wickham
\item
  \href{https://openintro-ims.netlify.app/}{Introduction to Modern
  Statistics} by Mine Çetinkaya-Rundel and Johanna Hardin
\item
  \href{https://www.tmwr.org/}{Tidy modeling with R} by Max Kuhn and
  Julia Silge
\item
  \href{https://bookdown.org/roback/bookdown-BeyondMLR/}{Beyond Multiple
  Linear Regression} by Paul Roback and Julie Legler
\end{itemize}

\subsection{Lectures}\label{lectures}

Lectures are designed to be interactive, so you gain experience applying
new concepts and learning from each other. My role as instructor is to
introduce you to new methods, tools, and techniques, but it is up to you
to take them and make use of them. A lot of what you do in this course
will involve writing code, and coding is a skill that is best learned by
doing. Therefore, as much as possible, you will be working on a variety
of tasks and activities throughout each lecture and lab. You are
expected to prepare for class by completing assigned readings, attend
all lecture sessions, and meaningfully contribute to in-class exercises
and discussion. Additionally, some lectures will feature application
exercises that will be graded based on completing what we do in class.

You are expected to bring a laptop, tablet, or Chromebook to each class
so that you can participate in the in-class exercises. Please make sure
your device is fully charged before you come to class, as the number of
outlets in the classroom may not be sufficient to accommodate everyone.

\subsection{Activities \& Assessment}\label{activities-assessment}

You will be assessed based on four components: application exercises,
homework, project, and oral exams.

\subsubsection{Application Exercises}\label{application-exercises}

Most lectures will have Application Exercises (AEs) that go along with
them. These exercises will give you an opportunity to practice applying
the statistical concepts and code introduced in the readings and
lectures. Typically, students who are present will receive full credit
on AEs and do not need to turn anything in. However, there will be times
where you must complete an AE outside of class and submit it. Students
who are late to class or miss class entirely must turn in a .qmd and
.pdf file for their AE before the start of the next lecture.
Specifically, AEs from Monday lectures are due Tuesday by 1:00pm MT.
Students with an excused absence will be graded for completion, students
without an excused absence will be graded for correctness.

\subsubsection{Homework}\label{homework}

In homework, you will apply what you've learned during lecture to
complete data analysis tasks. You may discuss homework assignments with
other students; however, homework should be completed and submitted
individually. Similar to lab assignments, homework must be typed up
using Quarto and submitted as .qmd and .pdf files in Canvas.

\subsubsection{Oral Exams}\label{oral-exams}

There will be two oral exams in this course. Each exam will include a
closed-notes component and an applied component which uses R. Through
these exams you have the opportunity to demonstrate what you've learned
in the course thus far. The exams will focus on both conceptual
understanding of the content and application through analysis and
computational tasks. The content of the exam will be related to the
content in reading assignments, lectures, application exercises, and
homework assignments. More detail about the exams will be given during
the semester. Please note the following course policies:

\begin{enumerate}
\def\labelenumi{\arabic{enumi}.}
\tightlist
\item
  You MUST receive at least an average of 60\% on your exams to pass the
  class.
\item
  Students will an average lower than 70\% may retake one of their oral
  exams but will be capped at 70\% for their exam grade.
\item
  \textbf{ANY UNCITED CODE ON YOUR HOMEWORK WHICH IS NOT COVERED IN
  CLASS IS ELIGIBLE FOR INCLUSION IN YOUR ORAL EXAM.}
\item
  The R used on the exams will be basic. The purpose of including R is
  to ensure you know the basic R functions we use regularly, R syntax,
  and your R ``workflow''. Do you know the basic functions? If you run
  into a (common) error, can you debug it yourself? Can you interpret
  the R Help menu? Do you understand common terminology
  (e.g.~``argument'', ``function'', ``output'')?
\item
  The specific questions asked during the Oral Exams will be different
  for every student and for every attempt, although similar in style and
  difficulty.
\end{enumerate}

\subsubsection{Project}\label{project}

The purpose of the \href{./project/project-instructions.qmd}{final
project} is to apply what you've learned throughout the semester to
analyze an interesting data-driven research question. The project will
be completed with your in pairs, and each team will present their work
through a written report and poster presentation taking place during the
last day of class. More information about the project will be provided
during the semester.

\subsection{Grading}\label{grading}

The final course grade will be calculated as follows:

\begin{longtable}[]{@{}ll@{}}
\toprule\noalign{}
Category & Percentage \\
\midrule\noalign{}
\endhead
\bottomrule\noalign{}
\endlastfoot
Homework & 25\% \\
Final Project & 25\% \\
Exam 01 & 20\% \\
Exam 02 & 20\% \\
Application Exercises & 10\% \\
\end{longtable}

\emph{Note: You must receive at least a 60\% on your two exams to pass
the course.}

The final letter grade will be determined based on the following
thresholds:

\begin{longtable}[]{@{}ll@{}}
\toprule\noalign{}
Letter Grade & Final Course Grade \\
\midrule\noalign{}
\endhead
\bottomrule\noalign{}
\endlastfoot
A & \textgreater= 93 \\
A- & 90 - 92.99 \\
B+ & 87 - 89.99 \\
B & 83 - 86.99 \\
B- & 80 - 82.99 \\
C+ & 77 - 79.99 \\
C & 73 - 76.99 \\
C- & 70 - 72.99 \\
D+ & 67 - 69.99 \\
D & 63 - 66.99 \\
D- & 60 - 62.99 \\
F & \textless{} 60 \\
\end{longtable}

\subsection{Five tips for success}\label{five-tips-for-success}

Your success in this course depends very much on you and the effort you
put into it. The course has been organized so that the burden of
learning is on you. I will help you by providing you with materials and
answering questions and setting a pace, but for this to work you must do
the following:

\begin{enumerate}
\def\labelenumi{\arabic{enumi}.}
\item
  Complete all the preparation work before class.
\item
  Ask questions. As often as you can. In class, out of class. Ask me,
  ask your friends, ask the person sitting next to you. This will help
  you more than anything else. If you get a question wrong on an
  assessment, ask why. If you're not sure about the homework, ask. If
  you hear something on the news that sounds related to what we
  discussed, ask. If the reading is confusing, ask.
\item
  Do the readings.
\item
  Do the homework. The earlier you start, the better. It's not enough to
  just mechanically plow through the exercises. You should ask yourself
  how these exercises relate to earlier material, and imagine how they
  might be changed (to make questions for an exam, for example.)
\item
  Don't procrastinate. The content builds upon what was taught in
  previous weeks, so if something is confusing to you on Day 2, Day 3
  will become more confusing, Day 4 even worse, etc. Don't let the week
  end with unanswered questions. But if you find yourself falling behind
  and not knowing where to begin asking, come to office hours and I can
  help you identify a good (re)starting point.
\end{enumerate}

\subsection{Course policies}\label{course-policies}

\subsubsection{Academic honesty}\label{academic-honesty}

\textbf{TL;DR: Don't cheat!}

\begin{itemize}
\item
  The homework assignments must be completed individually but you are
  welcome to discuss the assignment with classmates (e.g., discuss
  what's the best way for approaching a problem, what functions are
  useful for accomplishing a particular task, etc.). However you may not
  directly share (i.e.~via copy/paste or copying) answers to homework
  questions (including and especially any code) with anyone other than
  myself.
\item
  For the projects, collaboration within teams is not only allowed, but
  expected. Communication between teams at a high level is also allowed
  however you may not share code or components of the project across
  teams.
\item
  \textbf{Reusing code}: Unless explicitly stated otherwise, you may
  make use of online resources (e.g.~StackOverflow) for coding examples
  on assignments. If you directly use code from an outside source (or
  use it as inspiration), you must explicitly cite where you obtained
  the code. Any recycled code that is discovered and is not explicitly
  cited will be treated as plagiarism. Furthermore, \textbf{ANY UNCITED
  CODE OR CONTENT ON YOUR HOMEWORK WHICH IS NOT COVERED IN CLASS IS
  ELIGIBLE FOR INCLUSION IN YOUR ORAL EXAMS.}
\item
  \textbf{Use of artificial intelligence (AI)}: You should treat AI
  tools, such as ChatGPT, the same as other online resources. There are
  two guiding principles that govern how you can use AI in this
  course:\footnote{These guiding principles are based on
    \href{https://docs.google.com/document/d/1WpCeTyiWCPQ9MNCsFeKMDQLSTsg1oKfNIH6MzoSFXqQ/preview}{\emph{Course
    Policies related to ChatGPT and other AI Tools}} developed by Joel
    Gladd,
    Ph.D.\href{https://sta101-f23.github.io/course-syllabus.html\#fnref1}{↩︎}}
  (1) \emph{Cognitive dimension:} Working with AI should not reduce your
  ability to think clearly. We will practice using AI to
  facilitate---rather than hinder---learning. (2) \emph{Ethical
  dimension}\textbf{:} Students using AI should be transparent about
  their use and make sure it aligns with academic integrity.

  \begin{itemize}
  \item
    \textbf{AI tools for code:} You may make use of the technology for
    coding examples on assignments; if you do so, you must explicitly
    cite where you obtained the code. Any recycled code that is
    discovered and is not explicitly cited will be treated as
    plagiarism. You may use
    \href{https://guides.lib.monash.edu/c.php?g=219786&p=6972087}{these
    guidelines} for citing AI-generated content.
  \item
    \textbf{No AI tools for narrative:} Unless instructed otherwise, AI
    is \ul{\textbf{not}} permitted for writing narrative on assignments.
    In general, you may use AI as a resource as you complete assignments
    but not to answer the exercises for you. You are ultimately
    responsible for the work you turn in; it should reflect your
    understanding of the course content.
  \end{itemize}
\end{itemize}

If you are unsure if the use of a particular resource complies with the
academic honesty policy, please ask.

Regardless of course delivery format, it is the responsibility of all
students to understand and follow all College of Idaho policies,
including academic integrity (e.g., completing one's own work, following
proper citation of sources, adhering to guidance around group work
projects, and more). Ignoring these requirements is a violation of the
Honor Code.

\subsubsection{Late work policy}\label{late-work-policy}

The due dates for assignments are there to help you keep up with the
course material and to ensure the teaching team can provide feedback
within a timely manner. I understand that things come up periodically
that could make it difficult to submit an assignment by the deadline.

\begin{itemize}
\item
  \textbf{Late Homework:} There will be a 5\% deduction for each 24-hour
  period the assignment is late for the first two days. After 2 days,
  students will receive a 30\% reduction. No homework will be accepted
  after it is returned to the class
\item
  \textbf{Late Application Exercises:} AEs are due the day after the
  class they are assigned. No late work is accepted for application
  exercises, since these are designed as in-class activities to help you
  prepare for homework.
\item
  \textbf{School-Sponsored Events/Illness:} If an application exercise,
  exam, or project must be missed due to a school-sponsored event, you
  must let me know at least a week ahead of time so that we can schedule
  a time for you to make up the work before you leave. If you must miss
  a exam or a project presentation due to illness, you must let me know
  before class that day so that we can schedule a time for you to take a
  make-up quiz or exam. Failure to adhere to this policy will result in
  a 35\% penalty on the corresponding assignment.
\end{itemize}

\subsubsection{Regrade Requests}\label{regrade-requests}

Regrade requests must be submitted via email within a week of when an
assignment is returned. Regrade requests will be considered if there was
an error in the grade calculation or if you feel a correct answer was
mistakenly marked as incorrect. Requests to dispute the number of points
deducted for an incorrect response will not be considered. Note that by
submitting a regrade request, the entire question will be graded which
could potentially result in losing points.

\emph{No grades will be changed after the final project presentations.}

\subsection{Important dates}\label{important-dates}

\begin{itemize}
\tightlist
\item
  \textbf{Aug 27:} Classes begin
\item
  \textbf{Sep 1:} Labor Day - NO CLASS
\item
  \textbf{Sep 10:} Last day to drop
\item
  \textbf{Nov 20:} Last day to withdraw with W or elect Pass/Fail
\item
  \textbf{Nov 24-28:} Thanksgiving Holiday - NO CLASSES
\item
  \textbf{Dec 4:} Last Day of Classes
\end{itemize}




\end{document}
